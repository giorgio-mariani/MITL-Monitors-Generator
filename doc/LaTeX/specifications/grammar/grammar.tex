\documentclass[10pt,a4paper]{article}

%Import dei pacchetti necessari
\usepackage[T1]{fontenc}
\usepackage[utf8]{inputenc}
\usepackage[italian]{babel}
\usepackage[italian]{varioref}
\usepackage{amsfonts,amsmath,amssymb,graphicx}
\usepackage[boxed]{algorithm}
\usepackage[noend]{algorithmic}
\usepackage{listings}
\usepackage{syntax}
\usepackage{paracol}
\usepackage[hidelinks]{hyperref}



%command definitions
\newcommand{\Simulink}{\mbox{\emph{Simulink}\textsuperscript{\textregistered}}}
\newcommand{\Matlab}{\mbox{\emph{MATLAB}\textsuperscript{\textregistered}}}
\newcommand{\MathWorks}{\mbox{\emph{MathWorks}\textsuperscript{\textregistered}}}
\newcommand{\BLTL}{\mbox{\emph{BLTL}}}
\newcommand{\BLTLstar}{\mbox{\emph{BLTL*}}}
\newcommand{\Mex}{\mbox{\emph{mex}}}

\newcommand{\BlockType}[1]{\textbf{#1}}


%Notazione matematica
\newcommand{\Reals}{\mathbb{R}}
\newcommand{\Naturals}{\mathbb{N}}
\newcommand{\Integers}{\mathbb{Z}}

%Definizioni relative teoria BLTL
\newcommand{\Variables}{\mathcal{V}}
\newcommand{\Horizon}{h}
\newcommand{\StartTime}{s}
\newcommand{\MinTime}{minTime}
\newcommand{\MetaRel}{\mathcal{R}}
\newcommand{\Until}[1]{\mathbf{U^{#1}}}
\newcommand{\UntilUnit}[1]{\Delta^{#1}}
\newcommand{\Future}[1]{\Diamond^{#1}}
\newcommand{\Globally}[1]{\Box^{#1}}


%Definizioni Simulink
\newcommand{\Mstep}{\emph{Major-timestep}}

% RISORSE SISTEMA
\newcommand{\InputFile}{\emph{input file}}
\newcommand{\OutputDir}{\emph{output directory}}
\newcommand{\OutputLibrary}{\emph{output library}}

%Comandi grammatica
\newcommand{\Grammar}[1]{\textbf{#1}}
\newcommand{\FALSE}{\Grammar{FALSE}}
\newcommand{\TRUE}{\Grammar{TRUE}}
\newcommand{\NOT}{\Grammar{NOT}}
\newcommand{\AND}{\Grammar{AND}}
\newcommand{\OR}{\Grammar{OR}}

\newcommand{\UNTILword}{\Grammar{UNTIL}}
\newcommand{\FUTUREword}{\Grammar{FUTURE}}
\newcommand{\GLOBALLYword}{\Grammar{GLOBALLY}}

\newcommand{\UNTIL}[1]{\UNTILword[#1]}
\newcommand{\FUTURE}[1]{\FUTUREword[#1]}
\newcommand{\GLOBALLY}[1]{\GLOBALLYword[#1]}

% FONT TYPEFACE SETTINGS
\renewcommand*\rmdefault{ppl}

% PAGE LAYOUT SETTINGS
\usepackage{fullpage}

%INPUT FILE ENCODING
\newcommand{\InputFileEncoding}{\textbf{ISO/IEC 8859-1}}



%preamble
\author{Giorgio Mariani}
\date{}
\title{Definizione Formato File}

\begin{document}
	\maketitle
	\begin{abstract}
		In questo documento vengono specificate informazioni relative al file contenente le formule \BLTL{}.
	\end{abstract}
	 \begin{paracol}{2}
		\subsection*{Codifica file}
		Il file deve utilizzare la codifica \textbf{ISO/IEC 8859-1}, questa è un estensione della codifica \textbf{ASCII} pensata per codificare i caratteri utilizzati in Europa.
		
		\switchcolumn
		
		\subsection*{Formato file}
		Il file deve essere composto da una sequenza di formule \BLTL{} (La cui grammatica è definita più avanti) separate dalla stringa "\textbf{::}". Deve essere presente almeno una formula nel file. 
	\end{paracol}
	
	\section{Grammatica utilizzata}
	Consideriamo una formula in \emph{Bounded Linear Temporal Logic} (\BLTL{}) $\phi$, definiamo in base alla struttura di $\phi$ come questa debba essere rappresentata in forma di stringa nel file dato in input:
	
	\begin{itemize}
		\item $\phi=True$ oppure $\phi=False$: sono indicati rispettivamente attraverso le stringhe \TRUE{} e \FALSE{}.
		\item $\phi = \rho$, con:
			\begin{equation*}
				\rho = \sum_{i=0}^{n}{\alpha{}_i x_i\ \MetaRel\ \beta{}}
			\end{equation*}
			Viene indicato con la seguente sintassi:
			\begin{center}
				\newcommand{\Prod}{\textbf{*}}
				\newcommand{\Sum}{\textbf{+}}
				$\alpha_1$\Prod$x_1$ \Sum{} $\alpha_2$\Prod$x_2$ \emph{...} $\alpha_n$\Prod$x_n$ $\MetaRel$ $\beta$
			\end{center}
			
			Espressioni del tipo $\alpha_i$ e $x_i$ rappresentano rispettivamente un valore reale non negativo ed un nome di variabile.
			
			I nomi di variabile devono rispettare la seguente \emph{regular expression}\footnote{sintassi del programma \textbf{egrep}.}:
			\begin{center}
				[a-zA-Z\_][a-zA-Z0-9\_]*
			\end{center}
			
			Espressioni del tipo  $\alpha_i$*$x_i$ possono anche essere scritte come:  $\alpha_i$$x_i$, $x_i$*$\alpha_i$, $x_i$$\alpha_i$ oppure soltanto $x_i$ se il valore di $\alpha_i$ è $1$.
			
			Con \emph{...} abbiamo indicato una sequenza arbitrariamente lunga di espressioni del tipo $\alpha_i$*$x_i$ precedute dal carattere \textbf{+} oppure \textbf{-} (nel primo prodotto può essere omesso).
			
			La espressione $\MetaRel$ indica una tra seguenti stringhe: <, <=, >, >=, $\sim$=, =. La espressione $\beta$ è un valore reale.
			 
			 \newpage
			 \hrulefill
			 %-----------------------------------------------------------------------------------------------------
			 
			\begin{paracol}{3}
			\item $\phi = \lnot f_1$:\\ è indicata dalla parola chiave \NOT{} ed è  espressa come segue:
				\begin{center}
					\NOT{} \emph{formula}
				\end{center}
			
			\switchcolumn
			
			\item $\phi =  \Future{\alpha} f_1$:\\ è indicata dalla parola chiave \FUTUREword{} ed è  espressa come segue:
			\begin{center}
				\FUTURE{$\alpha$} \emph{formula}
			\end{center}
			Con $\alpha$ valore reale positivo.
			
			\switchcolumn
			
			\item $\phi =  \Globally{\alpha} f_1$:\\ è indicata dalla parola chiave \GLOBALLYword{} ed è  espressa come segue:
			\begin{center}
				\GLOBALLY{$\alpha$} \emph{formula}
			\end{center}
			Con $\alpha$ valore reale positivo.
		\end{paracol}
	
		I costrutti \NOT, \GLOBALLYword{} e \FUTUREword{} hanno la precedenza più alta, quindi nel caso si voglia utilizzarli per espressioni non banali occorre racchiudere il corpo dentro a delle parentesi tonde.
		
		\paragraph{ESEMPIO}
			\begin{center}
				\FUTURE{1}( \emph{formula} \AND{} \emph{formula} )
			\end{center}
		\hrulefill
		%-----------------------------------------------------------------------------------------------------
	
		\begin{paracol}{2}
			\item $\phi = f_1 \land f_2$: è indicata dalla parola chiave \AND{} ed è  espressa come segue:
				\begin{center}
					\emph{formula} \AND{} \emph{formula}
				\end{center}
				L'operatore \AND{} è il costrutto con la seconda precedenza più alta. La associatività dell'operatore è da sinistra verso destra.
		
		\switchcolumn
		
			\item $\phi = f_1 \lor f_2$: è indicata dalla parola chiave \OR{} ed è  espressa come segue:
				\begin{center}
					\emph{formula} \OR{} \emph{formula}
				\end{center}
				L'operatore \OR{} è il costrutto con la terza precedenza più alta. La associatività dell'operatore è da sinistra verso destra.
		\end{paracol}
		\hrulefill
		%-----------------------------------------------------------------------------------------------------
	
			
			\item $\phi = f_1 \Until{\alpha} f_2$:\\
			Un espressione dei questo tipo è indicata dalla parola chiave \UNTILword{} ed è  espressa come segue:
				\begin{center}
					\emph{formula} \UNTIL{$\alpha$} \emph{formula}
				\end{center}
				Con $\alpha$ valore reale positivo.
				L'operatore \UNTILword{} è il costrutto con la precedenza minore. La associatività dell'operatore è da sinistra verso destra.
	\end{itemize}

	
	
	
	\section{Dettagli grammatica}
	\subsection{Token utilizzati}
	I token non banali utilizzati per generare il linguaggio sono:
	\begin{description}
		\item[<NUMBER>] token utilizzato per rappresentare un valore reale non negativo. Le stringhe identificate da questo token sono quelle che rispettano la seguente \emph{regular expression}:
		\begin{center}
			[0-9]+(\textbackslash.[0-9]+)?(e(\textbackslash+|\textbackslash-)?[0-9]+)?	
		\end{center}
	
		\item[<VARIABLE>] token utilizzato per rappresentare un nome di variabile. Le stringhe identificate da questo token sono quelle che rispettano la seguente \emph{regular expression}:
		\begin{center}
				[a-zA-Z\_][a-zA-Z0-9\_]*	
		\end{center}
	
		\item[<SIGN>] token utilizzato per rappresentare un il segno + oppure il segno -.
		\item[<RELATION>]  token utilizzato per rappresentare i simboli di relazione: <, <=, >, >=, $\sim$=, =.
	\end{description}
	Gli spazi, tab, e i caratteri di nuova linea vengono tutti ignorati.
	
	\subsection{Regole di produzione}
	Regole \emph{BNF} utilizzate per generare la grammatica sopra descritta:
	\setlength{\grammarindent}{7em} 
	\begin{grammar}
	
	<until\_rule> ::= <until\_rule> `\UNTILword[' <NUMBER> `]' <or\_rule> \alt <or\_rule>
	
	<or\_rule> ::= <or\_rule> `\OR' <and\_rule> \alt <and\_rule>
	
	<and\_rule> ::= <and\_rule> `\AND' <unary\_rule> \alt <unary\_rule>
	
	<unary\_rule> ::= `\NOT'  <unary\_rule> \alt `\GLOBALLYword[' <NUMBER> `]' <unary\_rule> \alt  `\FUTUREword[' <NUMBER> `]' <unary\_rule>\alt  <pred\_rule>
	
	<pred\_rule> ::= <sign\_rule> <sum\_rule> <RELATION> <sign\_rule> <NUMBER> \alt `\TRUE' \alt `\FALSE' \alt `(' <until\_rule> `)'
	
	<sum\_rule> ::= <prod\_rule> <SIGN> <sum\_rule> \alt <prod\_rule>
	
	<prod\_rule> ::= <NUMBER> `*' <VARIABLE> \alt  <VARIABLE> `*' <NUMBER> \alt <NUMBER> <VARIABLE> \alt <VARIABLE> <NUMBER> \alt <VARIABLE>
	
	<sing\_rule> ::= <SIGN> \alt <empty string>
	\end{grammar}

	
	
\end{document}